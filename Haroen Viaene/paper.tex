\documentclass[12pt, a4paper]{paper}
\usepackage[T1]{fontenc}
\usepackage[utf8x]{inputenc}
\usepackage[parfill]{parskip}
\usepackage[dutch]{babel}
\usepackage[compact]{titlesec}
\usepackage{blindtext}
\usepackage{microtype,fullpage, graphicx}

\usepackage[scaled]{helvet}
\renewcommand\familydefault{\sfdefault}

\title{
\vspace{-2cm}
Paper ethisch onderwerp\\
\vspace{-1cm}
\hfill\includegraphics[height=1cm]{logo.pdf}}
\subtitle{`I cannot sit at home, and so I work in a rusthuis in Zemst'
\vspace{-.5cm}}
\author{
\vspace{-1cm}
\begin{large}
  Haroen Viaene
\end{large}
\vspace{-1cm}
}

\begin{document}

\maketitle

\begin{abstract}

Het artikel dat ik gekozen heb, komt uit DS Weekblad van 16 januari 2016 en gaat over de vluchteling Charbel Hermez die uit Syrië komt en nu voor dementerende bejaarden zorgt.

De schrijfster van dit artikel begint met te zeggen dat dit een vervolgartikel is op een die geschreven is op 24 December van vorig jaar. Hierin vertelt ze het verhaal van de moeder van Charbel, die in het noorden van Syrië woont. Haar jongste zoon (15) heeft vorig jaar het leven gelaten toen hij bij de Peshmerga (Koerdische verzetsgroep) voegde om een vuist te maken tegen Daesh.

Hoewel hij de raad kreeg van zowat iedereen dat hij te jong was om te vechten en dat hij er niet klaar voor was, drong hij toch aan om te vechten. Niet lang er na vond zijn familie foto's van zijn lijk op een extremistische website.

Na dat interview raadde de moeder van Charbel aan om te vragen hoe het met Charbel ging. Hij woont sinds oktober 2014 in Leuven en is sinds februari erkend als vluchteling.

% \Blindtext[2]

\end{abstract}

\section{Waarom heb ik dit artikel gekozen}

Dit artikel spitst zich op een unieke manier toe op een hele hoop problematiek die we momenteel hebben. Er wordt aangekaart over de oorlog die woedt in Syrië,

% \Blindtext[3]

\section{Mijn mening over dit onderwerp}



% \Blindtext[3]

\end{document}
