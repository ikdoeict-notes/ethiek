\documentclass[12pt, a4paper]{paper}
\usepackage[T1]{fontenc}
\usepackage[utf8x]{inputenc}
\usepackage[parfill]{parskip}
\usepackage[dutch]{babel}
\usepackage[compact]{titlesec}
\usepackage{blindtext}
\usepackage{microtype,fullpage, graphicx}

\usepackage[scaled]{helvet}
\renewcommand\familydefault{\sfdefault}

\title{
\vspace{-2cm}
Paper ethisch onderwerp\\
\vspace{-1cm}
\hfill\includegraphics[height=1cm]{logo.pdf}}
\subtitle{`I cannot sit at home, and so I work in a rusthuis in Zemst'
\vspace{-.5cm}}
\author{
\vspace{-1cm}
\begin{large}
  Haroen Viaene
\end{large}
\vspace{-1cm}
}

\begin{document}

\maketitle

\begin{abstract}

Het artikel dat ik gekozen heb, komt uit DS Weekblad van 16 januari 2016 en gaat over de vluchteling Charbel Hermez die uit Syrië komt en nu voor dementerende bejaarden zorgt.

De schrijfster van dit artikel begint met te zeggen dat dit een vervolgartikel is op een die geschreven is op 24 December van vorig jaar. Hierin vertelt ze het verhaal van de moeder van Charbel, die in het noorden van Syrië woont. Haar jongste zoon (15) heeft vorig jaar het leven gelaten toen hij bij de Peshmerga (Koerdische verzetsgroep) voegde om een vuist te maken tegen Daesh.

Hoewel hij de raad kreeg van zowat iedereen dat hij te jong was om te vechten en dat hij er niet klaar voor was, drong hij toch aan om te vechten. Niet lang er na vond zijn familie foto's van zijn lijk op een extremistische website.

Na dat interview raadde de moeder van Charbel aan om te vragen hoe het met Charbel ging. Hij woont sinds oktober 2014 in Leuven en is sinds februari erkend als vluchteling.

Sinds augustus werkt hij in een tehuis voor mensen met dementie, en voelt zich daar goed op zijn plaats. Hij plant om zijn cursussen Nederlands zo snel mogelijk af te werken en zo een job in de zorg te krijgen.

% \Blindtext[2]

\end{abstract}

\section{Waarom heb ik dit artikel gekozen}

Dit artikel spitst zich op een unieke manier toe op een hele hoop problematiek die we momenteel hebben. Er wordt aangekaart over de oorlog die woedt in Syrië, misdaden begaan door terroristische groeperingen, hoe jongeren omgaan met de situatie daar. Maar aan de andere kant gaat dit artikel over integratie, immigratie en de moeilijkheden die daarbij komen kijken.

We kunnen heel veel leren uit de zogenaamde ``succesvolle'' vluchtelingen en hoe die omgaan met het veranderen van context, om zo fouten in ons beleid aan te passen.

Ik heb een artikel over dit onderwerp gekozen omdat het niet alleen een heel \emph{hot} thema is, maar ook omdat het me nauw aan het hart ligt. Mijn moeder heeft een aantal jaar met vluchtelingen gewerkt in het Rode Kruiscentrum in Brugge en ikzelf heb op het einde van deze zomervakantie mijn best gedaan om hulp te verlenen in het Maximiliaanpark in Brussel (voor het WTC-complex) omdat we mee verantwoordelijk zijn voor wat er gebeurt.

% \Blindtext[3]

\section{Mijn mening over dit onderwerp}

Persoonlijk vind ik dat we ons zo veel mogelijk moeten openstellen voor mensen van een andere cultuur. Er is een hele hoop kennis die wij hebben en zij en niet, en ook omgekeerd.

Ik vind eigenlijk dat we nog niet ver genoeg gaan met mondialisering, we kunnen zo de krachten bundelen uit allerlei verschillende samenlevingen en dan een evenwichtige mix maken.

Uiteraard vind ik ook dat aan dat idee nadelen verbonden hangen. Het belangrijkste nadeel lijkt mij dat we moeten leren leven met minder in onze maatschappij. Als we de ongelijkheid proberen tegen te gaan om van middenklasse onze norm te maken in plaats van enkel bij ons, zal die klasse uiteraard gemiddeld zich minder kunnen permitteren.

Dit is slechts een lange-termijn-denkoefening van mij, maar het lijkt me desalniettemin interessant om iedereen een kans te geven het leven dat hij wil leven te lijden.

Vaak komt de kritiek op vluchtelingen uit Syrië dat dit geen arme mensen zijn, en dus ``per definitie'' geen vluchteling zijn. Persoonlijk vind ik het zeer fout om iemand die wel geld heeft te beledigen en minder kansen te geven omaat die er het geld voor over heeft te vluchten naar hier.

Er hangen ook een hele hoop negatieve aspecten aan het vluchten uit je thuisregio. Een allerbelangrijkste is dat je voor een hele tijd je familie niet meer kan zien, omdat het niet haalbaar is om met allemaal samen te vluchten. Dit hangt ook samen met het volgende probleem, namelijk dat het heel veel geld kost om te vluchten.

In dit artikel worden kosten van 11000 euro beschreven voor de vlucht van Charbel. Dat is een hele hoop geld, en op die manier begrijp je ook beter waarom enkel rijke families het zich kunnen veroorloven om weg te vluchten uit de greep van een oorlog.

Vaak wordt ervoor gekozen om de oudste zoon te sturen naar Europa om zo het proces te doorlopen en dan te opteren voor gezinshereniging. Het is veel te duur om allemaal samen te gaan.

Ik vind dat dit een van de grotere problemen is die opgelost moeten worden. Op dit moment kan slechts een heel klein percentage van de bevolking het zich permitteren om een beter leven op te zoeken, en iedereen verdient dat.

Op dit moment nemen we een passieve houding in, als iemand toekomt zullen we dan beslissen over zijn toekomst en die van de familie, maar het zou een openbaring zijn moesten we als samenleving (breder dan alleen België dan) actief mensen zoeken en de helpen om naar hier te komen. Zo komen we bij een evenwichtigere samenleving

% \Blindtext[3]

\end{document}
